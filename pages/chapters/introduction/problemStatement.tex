\section{Problem Statement}

Carbon emissions from maritime shipping have been identified as a major contributor to global greenhouse gas emissions, with the International Maritime Organization estimating that shipping is responsible for around 3\% of global CO2 emissions \autocite{king_anthony_2022}.
To address this issue, the shipping industry has set targets to reduce its carbon footprint, and governments and international organizations have introduced policies and regulations to encourage emissions reduction.

However, measuring and monitoring carbon emissions from maritime shipping can be challenging due to the complexity of the industry and the lack of reliable data.
The Energy Efficiency Operational Indicator (EEOI) and the Carbon Intensity Indicator (CII) have been proposed as two metrics to assess the carbon efficiency of ships and enable comparison between different vessels and fleets \autocite{ZHANG2019118223,CHUAH2023115348}.
However, there is a need to better understand the relationship between EEOI and carbon emissions, as well as to identify the factors that influence this metrics.

Therefore, the aim of this thesis is to conduct a big data analysis of carbon emissions in maritime shipping, using EEXI as the main metric. Specifically, the study will:

\begin{itemize}
    \item Calculate EEOI for a sample of vessels using real-world data on fuel consumption and other operational parameters.
    \item Analyze the relationship between EEOI, CII, and carbon emissions, using statistical methods and machine learning algorithms.
    \item Identify the factors that influence EEOI and CII, such as vessel age, size, speed, and route, and examine their impact on carbon emissions.
    \item Evaluate the usefulness of EEOI and CII as metrics for monitoring and reducing carbon emissions in maritime shipping, and recommend potential improvements to these metrics.
\end{itemize}


Overall, the findings of this thesis will contribute to a better understanding of the carbon efficiency of maritime shipping and inform the development of policies and strategies for emissions reduction in this sector.