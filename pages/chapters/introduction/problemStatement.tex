\section{Problem Statement}

Carbon emissions from maritime shipping have been identified as a major contributor to global greenhouse gas emissions, with the International Maritime Organization estimating that shipping is responsible for around 3\% of global CO2 emissions \autocite{king_anthony_2022}.
To address this issue, the shipping industry has set targets to reduce its carbon footprint, and governments and international organizations have introduced policies and regulations to encourage emissions reduction.

However, measuring and monitoring carbon emissions from maritime shipping can be challenging due to the complexity of the industry and the lack of reliable data.
The Energy Efficiency Operational Indicator (EEOI) and the Carbon Intensity Indicator (CII) have been proposed as two metrics to assess the carbon efficiency of ships and enable comparison between different vessels and fleets \autocite{ZHANG2019118223,CHUAH2023115348}.
However, there is a need to better understand the relationship between EEOI and carbon emissions, as well as to identify the factors that influence this metrics.

This thesis aims to address this problem by leveraging big data analysis and detailed ship characteristics data to provide a more precise and holistic view of maritime shipping's carbon footprint. 
By doing so, it seeks to enhance decision-making processes related to fuel optimization, retrofit measures, and environmental sustainability, contributing to a more responsible and efficient maritime industry.