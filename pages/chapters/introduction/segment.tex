\section{Segments}

The maritime shipping industry encompasses many more segments, this thesis will specifically focus on bulk carriers and tank carriers.

\subsection{Bulk Carriers}

Bulk carriers are specialized vessels designed to transport unpackaged bulk cargo such as grains, coal, ores, and cement. These vessels play a crucial role in global trade, ensuring the efficient and cost-effective movement of essential raw materials. Bulk carriers can be further categorized into several segments based on their size, design, and specific functionality:

\begin{itemize}
    \item \textbf{Supramax}: A class of bulk carriers typically ranging from 50,000 to 60,000 deadweight tons (DWT), offering flexibility in port access and cargo handling.
    \item \textbf{Aframax}: Medium-sized vessels with a DWT of approximately 80,000, often used for short to medium-haul routes.
    \item \textbf{Handysize}: Smaller vessels ranging from 15,000 to 35,000 DWT, ideal for accessing ports with size restrictions.
    \item \textbf{Handymax}: Slightly larger than Handysize, with a DWT ranging from 40,000 to 50,000.
    \item \textbf{Panamax}: Specifically designed to pass through the Panama Canal, Panamax bulk carriers have a maximum DWT of around 80,000.
    \item \textbf{Capesize}: The largest bulk carriers, often exceeding 100,000 DWT, are named for their inability to traverse the Cape of Good Hope or Cape Horn, requiring them to navigate around these capes.
\end{itemize}

\subsection{Tank Carriers}

Tank carriers, also known as tanker ships, are designed to transport liquids such as crude oil, petroleum products, and chemicals. These vessels play a vital role in the energy sector, connecting oil-producing regions with global markets. Tank carriers can be categorized into several segments, including:

\begin{itemize}
    \item \textbf{VLCC (Very Large Crude Carriers)}: Among the largest tank carriers, VLCCs typically have a DWT ranging from 200,000 to 320,000 tons.
    \item \textbf{VLGC (Very Large Gas Carriers)}: Designed to transport liquefied petroleum gas (LPG) and other gas products, with a capacity ranging from 70,000 to 85,000 cubic meters.
\end{itemize}