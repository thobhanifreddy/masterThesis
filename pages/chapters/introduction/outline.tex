\section{Report Outline}

\begin{figure}[ht]
    \centering
    \includegraphics[width=1\textwidth]{images/thesis_outline.png}
    \caption{Outline of the thesis}
    \label{fig:outline}
\end{figure}

\noindent Chapter 2: Litrature Review: This chapter covers the background information and literature review of the thesis.
this section covers comprehensive review of existing papers and research related to carbon emissions in maritime shipping.
The aim is to provide a comprehensive overview of the current state of research in this field and identify any gaps or opportunities for further exploration.


\noindent Chapter 3: Data Collection and Understanding:
In this chapter, the focus will be on gathering and understanding the data required to perform analysis to understand emissions in martime shipping.
Various data sources will be explored, including industry databases, research publications, and government reports.
The aim is to acquire a comprehensive dataset that covers different aspects of carbon emissions in the maritime sector.
Additionally, this chapter will delve into the intricacies of the collected data, understanding its structure, variables, and potential limitations.

\noindent Chapter 4: Data Cleaning and Preprocessing:
Before conducting any data analysis, it is crucial to ensure the quality and integrity of the dataset.
This chapter will discuss the steps involved in cleaning and preprocessing the data.
This process may involve handling missing values, dealing with outliers, standardizing formats, and resolving inconsistencies.
By performing these necessary data cleaning procedures, the dataset will be prepared for further analysis, ensuring reliable and accurate results.

\noindent Chapter 5: Data Analysis Techniques:
In this chapter, various data analysis techniques specific to big data will be explored and applied to the cleaned dataset.
These techniques may include statistical analysis, machine learning algorithms, and data visualization methods.
The goal is to extract meaningful insights and patterns from the data to gain a comprehensive understanding of carbon emissions in maritime shipping.
Additionally, this chapter will discuss the tools and technologies utilized for data analysis and highlight any specific challenges encountered during the process.

\noindent Chapter 6: Evaluation of Results:
After performing the data analysis, this chapter will focus on evaluating and interpreting the obtained results.
The findings will be compared against existing literature, industry benchmarks, and regulatory standards to assess the significance and implications of the analysis.
The strengths and limitations of the analysis approach will be discussed, and recommendations for future research or practical applications will be provided.
This chapter aims to provide a comprehensive evaluation of the insights gained from the data analysis and their potential impact on the maritime shipping industry.

\noindent Conclusion:
The conclusion chapter will summarize the key findings and contributions of the thesis.
It will highlight the significance of the conducted big data analysis on emissions in maritime shipping and its implications for sustainability and environmental initiatives.
The conclusion will also discuss any potential limitations or challenges encountered during the research and suggest avenues for further exploration in this field.