\chapter{Introduction}
\setcounter{page}{1}

\section{Background and Motivation}

In the 21st century, Climate change is the biggest challenge faced by humanity. It poses a substantial danger to the survival of the inhabitants of our planet.
Human activities such as deforestation and burning of fossil fuels have led to a rise in global temperatures.
Becuase of this rise, there has been a rise in sea levels, extreme weather events, and loss of biodiversity.
There is an urgent need to reduce greenhouse gas emissions and transition to a sustainable, low-carbon future.

Maritime is essential to the global economy, transporting 90\% of the world's goods by volume.
It is also a major source of greenhouse gas emissions, with the International Maritime Organization (IMO)
estimating that maritime shipping accounts for 3\% of global carbon dioxide emissions.
While 3\% may seem small, it is important to note that this is a rapidly growing sector.
Without action, maritime shipping contribution to carbon emissions can increase upto 10-13\% in the next few decades..
Due to this fact, there is a growing global effort to reduce emissions from this sector. \autocite{king_anthony_2022}.

In accordance with sustainable Development Goal 13, in 2018, the inital stratergy was adopted by IMO's Environmental Protection Committee (MEPC),
during its 72nd session at IMO Headquarters in London, United Kingdom. Accorging to this stratergy,
the IMO will work towards reducing the total annual greenhouse gas emissions from international shipping by at least 50\% by 2050 compared to 2008 \autocite{imo-2018}.
In 76th ssssion MEPC in 2021, serval mandatory measures were adopted to reduce greenhouse gas emissions from international shipping,
which will help in achieving the goal of reducing emissions by 50\% by 2050 \autocite{imo-2021}. One of the important measures is the carbon intensity indicator (CII).

Maritime shipping is a complex and highly volatile system, generating very large data sets.
Big data analytics can be used to understand the complex system and make informed decisions.
It can facilitate operations such as monitoring of emission and predictive analysis of vessel performance.
This can help in reducing emissions and improving the efficiency of the maritime sector \autocite{ZAMAN2017537}.


\section{Big Data Analysis}

Big data analytics is where advanced analytic techniques operate on big data sets. Hence, big data
analytics is really about two things — \textit{big data} and \textit{analytics}.



\subsection{Big Data}

As the name suggests, big data is a large amount of data. There are other important attributes of big data. These are:  data variety and data velocity.

Thus we can define big data using 3 V's: \textit{volume}, \textit{variety}, and \textit{velocity} as showin in figure \ref*{bigData}.

\begin{figure}[h]
    \centering
    \includegraphics[width=0.5\textwidth]{images/big_data.png}
    \caption{Big Data: 3 V's \autocite{3vbigdata}}
    \label{bigData}
\end{figure}

Beyond these three V's, Big Data is also about how complicated the computing
problem is. Given the number of variables and number of data points for analysing the maritime shipping data. It is a very complicated problem.
Thus, in addition to the three V's identified by IBM, it would also be necessary to take complexity into account as shown in figure \ref*{bigDataComplex} \autocite{pence2014big}.

\begin{figure}[h]
    \centering
    \includegraphics[width=0.5\textwidth]{images/big_data_complex.png}
    \caption{Big Data: Beyong 3 V's - volume, velocity,
        variety, and complexity}
    \label{bigDataComplex}
\end{figure}


\subsection{What is Big Data Analytics?}

Big data analytics is the process of examining large and varied data sets to uncover hidden patterns, unknown correlations, market trends, customer preferences and other useful information that can help organizations make more-informed business decisions.
\section{Problem Statement}

Carbon emissions from maritime shipping have been identified as a major contributor to global greenhouse gas emissions, with the International Maritime Organization estimating that shipping is responsible for around 3\% of global CO2 emissions \autocite{king_anthony_2022}.
To address this issue, the shipping industry has set targets to reduce its carbon footprint, and governments and international organizations have introduced policies and regulations to encourage emissions reduction.

However, measuring and monitoring carbon emissions from maritime shipping can be challenging due to the complexity of the industry and the lack of reliable data.
The Energy Efficiency Operational Indicator (EEOI) and the Carbon Intensity Indicator (CII) have been proposed as two metrics to assess the carbon efficiency of ships and enable comparison between different vessels and fleets \autocite{ZHANG2019118223,CHUAH2023115348}.
However, there is a need to better understand the relationship between EEOI and carbon emissions, as well as to identify the factors that influence this metrics.

This thesis aims to address this problem by leveraging big data analysis and detailed ship characteristics data to provide a more precise and holistic view of maritime shipping's carbon footprint. 
By doing so, it seeks to enhance decision-making processes related to fuel optimization, retrofit measures, and environmental sustainability, contributing to a more responsible and efficient maritime industry.
\section{Research Question}

This theis will focus on answering following research questions:

\begin{enumerate}
    \item What is the relationship between vessel age and carbon emissions in maritime shipping?
    \item How do shipping routes affect carbon emissions in maritime shipping?
    \item What role do fuel types and engine technologies play in carbon emissions in maritime shipping?
    \item How can EEXI and CII be used to monitor and reduce carbon emissions in maritime shipping?
\end{enumerate}
\section{Report Outline}

\begin{figure}[ht]
    \centering
    \includegraphics[width=1\textwidth]{images/thesis_outline.png}
    \caption{Outline of the thesis}
    \label{fig:outline}
\end{figure}

\noindent Chapter 2: Litrature Review: This chapter covers the background information and literature review of the thesis.
this section covers comprehensive review of existing papers and research related to carbon emissions in maritime shipping.
The aim is to provide a comprehensive overview of the current state of research in this field and identify any gaps or opportunities for further exploration.


\noindent Chapter 3: Data Collection and Understanding:
In this chapter, the focus will be on gathering and understanding the data required to perform analysis to understand emissions in martime shipping.
Various data sources will be explored, including industry databases, research publications, and government reports.
The aim is to acquire a comprehensive dataset that covers different aspects of carbon emissions in the maritime sector.
Additionally, this chapter will delve into the intricacies of the collected data, understanding its structure, variables, and potential limitations.

\noindent Chapter 4: Data Cleaning and Preprocessing:
Before conducting any data analysis, it is crucial to ensure the quality and integrity of the dataset.
This chapter will discuss the steps involved in cleaning and preprocessing the data.
This process may involve handling missing values, dealing with outliers, standardizing formats, and resolving inconsistencies.
By performing these necessary data cleaning procedures, the dataset will be prepared for further analysis, ensuring reliable and accurate results.

\noindent Chapter 5: Data Analysis Techniques:
In this chapter, various data analysis techniques specific to big data will be explored and applied to the cleaned dataset.
These techniques may include statistical analysis, machine learning algorithms, and data visualization methods.
The goal is to extract meaningful insights and patterns from the data to gain a comprehensive understanding of carbon emissions in maritime shipping.
Additionally, this chapter will discuss the tools and technologies utilized for data analysis and highlight any specific challenges encountered during the process.

\noindent Chapter 6: Evaluation of Results:
After performing the data analysis, this chapter will focus on evaluating and interpreting the obtained results.
The findings will be compared against existing literature, industry benchmarks, and regulatory standards to assess the significance and implications of the analysis.
The strengths and limitations of the analysis approach will be discussed, and recommendations for future research or practical applications will be provided.
This chapter aims to provide a comprehensive evaluation of the insights gained from the data analysis and their potential impact on the maritime shipping industry.

\noindent Conclusion:
The conclusion chapter will summarize the key findings and contributions of the thesis.
It will highlight the significance of the conducted big data analysis on emissions in maritime shipping and its implications for sustainability and environmental initiatives.
The conclusion will also discuss any potential limitations or challenges encountered during the research and suggest avenues for further exploration in this field.