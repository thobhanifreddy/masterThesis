\chapter*{Abstract}
\addcontentsline{toc}{chapter}{Abstract}
\setlength{\parskip}{1em}

Maritime shipping stands as a pivotal pillar within the global economy, 
facilitating the transportation of over 90\% of the world's commodities and also major contributor to carbon emissions.
Maritime industy is a very volatile system  generating very large and complex data sets.
Employing the robust framework of Big Data analysis, this thesis explores the carbon emissions of maritime shipping through indicators like Carbon Intensity Indicator (CII) and Energy Efficiency Operating Indicator (EEOI).
Analysing emissions of about 9400 vessels grouped by segments and assigning them A to E grade based on their emissions helps us to understand complexity of emissions in maritime shipping with simple grading system.


The projection of Carbon Intensity Indicator (CII) trends for the period 2023 to 2026 elucidates a concerning pattern: a growing number of vessels may soon be graded D or E due to elevated carbon intensity.
The examination of the interplay between vessel speed, EEOI, and grade exposes noteworthy insights. It is evident that vessels exhibiting lower speed and reduced EEOI tend to achieve superior grades.
This revelation helps to inovate to find practical and cost-effective pathways for emissions mitigation, 
including measures such as the reduction of vessel speed, the implementation of low friction coatings, and the diligent maintenance of propellers and hulls.


This thesis makes significant strides in advancing our comprehension of maritime shipping's carbon impact. 
By offering pragmatic solutions, it not only addresses a pressing concern but also lays the groundwork for responsible and informed industry practices.