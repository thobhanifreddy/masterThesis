\chapter{Conclusion}

Initially, dividing the vessels into segments and assigning Grade from A to E for given year as Carbon intensity indicator provides a clear picture of the carbon emissions of the vessel.
Through the analysis of more than 9000 vessels, it was found that the average grade of vessels has been decreasing over the years. 
This shows that the carbon emissions of vessels has been increasing over the years.
If the emissions during the voyage is not reduced, Carbon intensity indicator of more and more vessels will fall under D or E grade. 

The thesis also shows that one of the key factor effecting the carbon emissions is the speed and EEOI of the vessel.
Using alternative fuels, applying low friction paint, voyage route optimization, and using latest hardware are some of the practical solutions to reduce the carbon emissions of the vessels


This study delved into a comprehensive investigation of carbon emissions in maritime shipping, a crucial yet intricate facet of the industry. Through the course of this research, several key questions were addressed:

\begin{enumerate}
    \item \textbf{Understanding Complexity with Simplicity}: The analysis demonstrated that even the multifaceted nature of emissions could be understood through a well-structured and comprehensible system. By breaking down complex elements, the thesis facilitated a clearer grasp of emissions dynamics.
    \item \textbf{Categorization of Vessels}: The categorization of vessels into distinct segments such as Capesize, Panamax, Supramax, and others enabled a more nuanced examination of carbon emissions. This segmentation provided unique insights into each category's characteristics and emissions behavior.
    \item \textbf{Factors Affecting Carbon Emissions}: A comprehensive analysis was conducted to identify the multitude of factors affecting a vessel's carbon emissions. This understanding forms a foundation for more targeted and effective emissions control strategies.
    \item \textbf{Realistic Emissions Reduction}: The thesis explored various cost-effective and practical measures to reduce carbon emissions. These findings not only contribute to environmental sustainability but also offer realistic paths for industry implementation.
\end{enumerate}

This thesis set the stage for future research that continues to push the boundaries of our understanding and capabilities. 
Looking ahead, there are several promising avenues for future research. 
The consideration of port emissions alongside voyage emissions presents an opportunity for a more holistic view of maritime carbon footprints. 
Additionally, analyzing more vessel segments will further refine our understanding and provide more detailed insights. 
These future efforts align with the ongoing pursuit of environmental responsibility and efficiency in the maritime shipping industry.

In conclusion, this thesis has made a significant contribution to the field by unraveling the complexities of maritime shipping's carbon emissions. 
It has provided valuable insights, practical solutions, and set the stage for future research that continues to push the boundaries of our understanding and capabilities.

